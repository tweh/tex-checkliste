\documentclass[
   ngerman,
   fontsize = 11.5pt,
   parskip = half,
   headings = small,
]{scrartcl}

% Basispakete
\usepackage[utf8]{inputenc}
\usepackage[T1]{fontenc}

% Sprache
\usepackage{babel}
\usepackage{csquotes}

% Schrift und Farbe
\usepackage[default]{sourcesanspro}
\usepackage[scale = 0.95]{sourcecodepro}
\usepackage{microtype}
\usepackage{metalogo}
   \setlogokern{La}{-0.185em}
   \setlogokern{aT}{-0.1em}
   \setlogokern{Te}{-0.12em}
   \setlogokern{eX}{-0.01em}
   \setLaTeXa{\scshape a}
   \setlogodrop[TeX]{0.4ex}
\usepackage{xcolor}
   \definecolor{spot}{RGB}{0,100,138}
   \definecolor{gray}{RGB}{199,215,221}
   \colorlet{gray}{spot!17}

% Layout
\usepackage{geometry}
   \geometry{%
      margin = 30mm,
   }
\usepackage{ragged2e}
\pagestyle{empty}
\setcounter{secnumdepth}{-1}

% Links
\usepackage[
   colorlinks,
   allcolors = spot
]{hyperref}
\urlstyle{same}

% Überschriften
\let\section\minisec

% Checklisten-Umgebung
\usepackage{enumitem}
   \newcommand{\checkbox}{\textcolor{gray}{\rule[-0.08em]{0.85em}{0.85em}}}
   \newlist{checklist}{itemize}{2}
   \setlist[checklist]{
      label = \checkbox,
      align = left,
      labelsep =    0.5em,
      %             +
      labelwidth =  1.0em,
      %             =
      leftmargin =  1.5em,
   }

% Befehl für Links
\newcommand{\hint}[1]{{%
   \par
   \vspace{-\parskip}%
   \footnotesize
   #1
}}

% Befehle für Code, Macronamen, Umgebungen, Programme, Pakete etc.
\newcommand{\code}[1]{\texttt{#1}}
\newcommand{\cs}[1]{\code{\textbackslash#1}}
\newcommand{\env}[1]{\code{\{#1\}}}
\newcommand{\prg}[1]{\hyperlink{https://ctan.org/pkg/#1}{\code{\itshape#1}}}
\newcommand{\pkg}[1]{\code{#1}}
\newcommand{\msg}[1]{\textit{\enquote{#1}}}
\newcommand{\sfx}[1]{\code{.#1}}

\begin{document}
\RaggedRight

% Titel/Kopf
\begin{center}
   \setlength{\parskip}{0pt}
   \bfseries
   \textls[50]{CHECKLISTE}
   
   \smallskip
   \LARGE\color{spot}
   Qualität von Code und Daten in \LaTeX-Projekten
   
   \vspace{\baselineskip}
   \normalsize\mdseries\normalcolor
   Zusammengestellt von Tobi W\_ (\url{mail@tobiw.de})\\
   Version 0.0 --- \url{https://github.com/tweh/tex-checkliste}\\
\end{center}

\vspace{\baselineskip}
{\footnotesize\itshape
   Ohne Anspruch auf Vollständigkeit! Ergänzungen und Korrekturen gerne per
   E-Mail oder über GitHub.
\par}

\vspace{2\baselineskip}
\section{Allgemeines}
\begin{checklist}
   \item Sind alle Dateien einheitlich kodiert?
      \hint{In der Regel Unicode (UTF-8) oder ISO-8859-1 (latin1).}
   \item Lässt sich das Dokument fehlerfrei kompilieren?
   \begin{checklist}
      \item Sind alle verwendeten Befehle/Umgebungen definiert?
      \item Werden Umgebungen, Argumente, Gruppen etc. korrekt geschlossen?
      \item Sind alle eingebundenen \TeX-Dateien vorhanden?
      \item Sind alle eingebundenen Bilddateien vorhanden?
      \item \dots
   \end{checklist}
   \item Ist der Code bzw. das Projekt ordentlich angelegt?
   \begin{checklist}
      \item Sind lange Dokumente in einzelne Dateien aufgeteilt?
         \hint{\url{https://tex.stackexchange.com/q/246/4918}}
      \item Wurden Kommentare zur Erklärung/Beschreibung verwendet?
      \item Werden Einrückungen etc. verwendet, um den Code zu gliedern/strukturieren?
         \hint{\url{https://tex.stackexchange.com/q/40775/4918}}
   \end{checklist}
   \item Werden veraltete/obsolete Pakete oder Befehle etc. benutzt?
      \hint{\url{https://ctan.org/pkg/l2tabu}}
\end{checklist}

\section{Schrift und Text}
\begin{checklist}
   \item Sind alle verwendeten Schriften und Schriftschnitte verfügbar?
      \hint{Warnung \msg{Font shape `x' undefined} in der LOG-Datei prüfen.}
   \item Werden noch obsolete Schriftschalter (z.\,B. \cs{bf}) benutzt?
      \hint{\url{https://ctan.org/pkg/l2tabu}}
   \item Werden Absätze korrekt ausgezeichnet (Leerzeile oder \cs{par}; nicht \cs{\textbackslash})?
   \item Wird \cs{emph} (statt \cs{textit}) für einfache Auszeichnungen benutzt?
   \item Werden Auszeichnungen im Text durch geeignete Befehle (ggf. selbst definiert) erreicht?
      \hint{Nur mit einem \emph{logischen Markup} bleibt das Dokument auch bei Änderungen
      an der Art der Auszeichnung konsistent.}
   \item Werden Anführungszeichen korrekt gesetzt?
      \hint{Am besten mit \pkg{csquotes}.}
   \item Wurden Abkürzungen mit schmalen Leerzeichen gesetzt?
      \hint{\code{z.\textbackslash,B.} statt \code{z. B.} oder \code{z.B.}}
   \item Wurden bereits viele manuelle Umbrüche u.\,ä. eingesetzt, die beim Satz in einem
      ggf. anderen Layout stören?
\end{checklist}

\section{Mathematiksatz und Formeln}
\begin{checklist}
   \item Wird der Mathe-Modus korrekt eingesetzt und alles entsprechend gekennzeichnet?
      \hint{Bspw. müssen auch einzelne Variablen im Fließtext ausgezeichnet sein.}
   \item Werden noch die veralteten Matheumgebungen \code{\$\$...\$\$} oder \env{eqnarray}
      benutzt?
      \hint{\url{http://texblog.net/latex-archive/maths/eqnarray-align-environment/}}
      \hint{\url{https://tex.stackexchange.com/q/503/4918}}
   \item Werden Zahlen und Einheiten korrekt formatiert?
      \hint{Am besten mit \pkg{siunitx}.}
\end{checklist}

\section{Verweise, Quellen und Co.}
\begin{checklist}
   \item Werden Querverweise mit der \cs{label}-\cs{ref}-Mechanik erzeugt?
   \item Sind alle Querverweise eindeutig definiert?
      \hint{Warnungen \msg{Reference `x' on page n undefined} oder \msg{Label `x' multiply defined} in der LOG-Datei prüfen und im Dokument nach \enquote{\textbf{??}} suchen.}
   \item Werden Literaturverzeichnis und -verweise automatisch erzeugt?
   \item Sind alle Literaturverweise (Zitierschlüssel) definiert?
      \hint{Warnung \msg{Citation `x' on page n undefined} in der LOG-Datei prüfen.}
   \item Lässt sich die Bibliografie ohne Fehler und Warnungen verarbeiten?
      \hint{Warnungen und Fehlermeldungen des Bibliographieprogramms (z.\,B. \prg{biber}) beachten.}
   \item Lassen sich Index- und Glossarhilfsdateien ohne Fehler und Warnungen verarbeiten?
      \hint{Warnungen und Fehlermeldungen des entsprechenden Programms (z.\,B. \prg{xindy}) beachten.}
\end{checklist}

\section{Abbildungen und Tabellen}
\begin{checklist}
   \item Sind eingebundene Bilddateien vorhanden (s.\,o.)?
   \item Haben eingebundene Bilddateien eine ausreichende Auflösung?
      \hint{Kompilierte PDF bspw. mit Preflight-Funktion von Acrobat prüfen.}
   \item Werden Gleitumgebungen wie \env{figure} oder \env{table} (sinnvoll) benutzt?
   \item Sind Abbildungen und Tabellen mit einer Beschriftung (\cs{caption}) versehen?
\end{checklist}

\end{document}